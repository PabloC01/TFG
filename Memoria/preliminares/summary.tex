% !TeX root = ../libro.tex
% !TeX encoding = utf8
%
%*******************************************************
% Summary
%*******************************************************

\selectlanguage{english}
\chapter{Summary}

Graph theory is present in various aspects of the modern world, such as the analysis of increasingly prominent social networks. It involves analyzing the different properties of these social networks through their representation as graphs. It is also present in other branches of science, such as chemistry, where it is used to study the chemical structure of different molecules. \\

This theory is also prevalent in crucial applications such as GPS or route calculation, which use multiple graphs as a base to calculate different routes according to specific problem requirements. \\ 

As we can observe, graph theory is very useful in modern life thanks to its diverse applications. In this work, we will focus on the search for paths of minimum length on graphs, that is, the paths between nodes with the shortest possible length within a graph. These types of paths are truly valuable. For instance, they allow us to establish bus lines in a city in a way that optimally covers the stops, using the shortest routes between them. \\

The main objective of this work is to develop a suitable theoretical framework to formalize the algorithms used in calculating paths of minimum length, as well as to test their correctness. The algorithmic complexity of these algorithms will also be studied. \\

In addition to studying the necessary mathematical foundations to understand and verify the correctness of these algorithms, some of the basic algorithms have been modified to calculate not just a single path of minimum length between two nodes, but to calculate all of them. This is because there may be multiple paths of minimum length, and knowing all of them can be highly useful in certain real-life problems. \\

This work will be structured into five different chapters. Firstly, a series of preliminary concepts will be discussed to establish the basic structures in graph theory, such as metric spaces and the distances associated with these spaces. \\

The second chapter will introduce the graph theory, providing a series of basic definitions to understand the concept of a graph itself. It will then study various properties associated with graphs and paths of minimum length, which will be essential to verify the correctness of certain algorithms. \\

In the third chapter, we will study the algorithms themselves. Firstly, we will provide a general idea of how the algorithm works, considering it from the perspective of metric spaces and proving its correctness. Finally, we will present the pseudocode associated with the algorithm and provide examples of its operation. \\

In the fourth chapter, we will study the algorithmic complexity of the algorithms. We will begin by establishing a suitable theoretical framework for the study of algorithmic complexity. This section is crucial as it allows us to determine beforehand whether an algorithm is capable of solving a specific problem of a certain size within a reasonable time frame or if it is impractical. \\

In the fifth chapter, we will discuss the implementation of the algorithms in C++ and the calculation of execution times to compare them with the complexity orders obtained in the fourth chapter. \\

In the final chapter, we will present the conclusions drawn from this work and discuss potential future research directions that can be derived from it.

% Al finalizar el resumen en inglés, volvemos a seleccionar el idioma español para el documento
\selectlanguage{spanish} 
\endinput
